\documentclass[11pt]{article}
\usepackage{amsmath}
\usepackage{amsbsy}
\usepackage{amssymb}
\usepackage{amsthm,amsfonts}
\usepackage{epsf,epsfig}
\usepackage{times}
\usepackage{mathptmx}
\usepackage{color}
\usepackage{nomelabel}
\usepackage[english]{babel}
\usepackage{srcltx}
\usepackage[boxed]{algorithm2e}
\usepackage{ifpdf}
%
% Standard definitions...
%
\input{def}
%
% True document
%
\newcommand{\muth}{\mu_\theta}
\begin{document}
%\maketitle
\begin{abstract}
  Due appunti in croce su beamforming \& C.
\end{abstract}

\section{Caso semplice}
\label{sect:3;appunti}

Si consideri lo schema in \fref{ref}.  Abbiamo una sorgente
monocromatica che emette $t \mapsto \exp(j2\pi ft)$ a distanza $R$
dall'origine e angolo (con la verticale) $\alpha$.  Supponiamo di
avere un microfono in posizione $(x,y)$.  Vogliamo determinare il
segnale acquisito $s : \reali \to \complessi$.

\begin{figure}
\centerline{\unafigura{figura-beamforming-2}}
\caption{
\label{fig:ref}}
\end{figure}
%

Sia $r$ la distanza tra il microfono e la sorgente. Sia
$d=\sqrt{x^2+y^2}$ la distanza del microfono dall'origine e sia
$\beta$ tale che $\tan\beta=-y/x$ (vedi figura). Usando il teorema 
del coseno
%
\begin{equation}
\label{eq:9;appunti}
\begin{split}
  r &= \sqrt{R^2 + d^2 - 2Rd \cos(\pi/2-\alpha+\beta)}  \\
    &= \sqrt{R^2 + d^2 - 2Rd \sin(\alpha-\beta)}  \\
    &= R \sqrt{1 + (d/R)^2 - 2(d/R) \sin(\alpha-\beta)}  
\end{split}
\end{equation}
%
Osserviamo ora che tipicamente $R \gg d$ ($R$ \`e dell'ordine dei
metri, $d$ dei centimetri).  Possiamo quindi trascurare $(d/R)^2$ (che
\`e dell'ordine di $10^{-4}$) rispetto a $d/R$.  Possiamo inoltre
approssimare la radice quadrata col suo sviluppo di Taylor intorno a 1
arrestato al primo termine ($\sqrt{1+x} \approx 1 + x/2$) ottenendo
%
\begin{equation}
\label{eq:10;appunti}
\begin{split}
  r &= R \sqrt{1 + (d/R)^2 - 2(d/R) \sin(\alpha-\beta)}  \\
  &\approx R \sqrt{1 - 2(d/R) \sin(\alpha-\beta)}  \\
  &\approx R (1 - (d/R) \sin(\alpha-\beta))  \\
  &= R - d \sin(\alpha-\beta) \\
  &= R - d \sin\alpha\cos\beta + d\sin\beta\cos\alpha \\
  &= R - x \sin\alpha - y \cos\alpha 
\end{split}
\end{equation}
%
dove abbiamo sfruttato una nota uguaglianza trigonometrica e il fatto
che $y=-d \sin\beta$, $x=d\cos\beta$.

Il segnale acquisito dal microfono all'istante $t$ ha lasciato la
sorgente $r/v$ istanti prima.  Ne segue che
%
\begin{equation}
\label{eq:11;appunti}
\begin{split}
  s(t) &= \exp\left(j 2\pi f (t-r/v)\right) \\
       &= \exp\left[j 2\pi f
  \left(t-\frac{R-x\sin\alpha-y\cos\alpha}{v} \right)\right] \\
       &= \exp(j 2\pi f t) \exp(-j 2\pi f R/v) 
  \exp\left[j2\pi f \left(\sin\alpha \frac{x}{v}+\cos\alpha\frac{y}{v}
    \right)\right] \\ 
       &= \exp(j 2\pi f t) \exp(-j 2\pi f R/v) 
  \exp\left[j2\pi f \left(\tau_x \sin\alpha+\tau_y \cos\alpha
    \right)\right] \\ 
       &= \commenta{\exp(j 2\pi f t) \exp(-j 2\pi f R/v)}{E(t)}
  \exp\left[j2\pi  \left(\frac{x}{\lambda}
    \sin\alpha+\frac{y}{\lambda}  \cos\alpha
    \right)\right] \\ 
\end{split}
\end{equation}
%
dove $\lambda$ \`e la lunghezza d'onda, $\tau_x=x/v$ \`e il tempo che
l'onda ci mette a percorrere una distanza $x$, simile per $\tau_y$.
Si noti che $E(t)$ \`e una versione ritardata dell'onda emessa dalla
sorgente e non dipende dalla posizione del microfono.

Supponiamo ora di avere un reticolo di microfoni disposti nelle
posizioni $(n\Delta_x, m\Delta_y)$ con $n,m\in\interi$ e di
``mischiare'' i contributi dei microfoni pesando il segnale acquisito
dal microfono $(n,m)$ con $a_{n,m}\in \complessi$.  Sia $T_x =
\Delta_x/v$, $T_y=\Delta_y/v$. Il risultato \`e
%
\begin{equation}
\label{eq:12;appunti}
\begin{split}
  x(t) &= \sum_{n,m\in \interi} a_{n,m} s_{n,m}(t) \\
  &= E(t) \sum_{n,m\in \interi} a_{n,m}
    \exp\left[j2\pi f \left(\frac{n\Delta_x}{v}
      \sin\alpha+\frac{m\Delta_y}{v} \cos\alpha 
    \right)\right] \\ 
  &= E(t) \sum_{n,m\in \interi} a_{n,m}
    \exp\left[j2\pi \left(n (f T_x)
      \sin\alpha+m (f T_y) \cos\alpha 
    \right)\right] \\ 
  &= E(t) \sum_{n,m\in \interi} a_{n,m}
    \exp\left(j2\pi [-f T_x \sin\alpha,  -f T_y \cos\alpha] [n,m]^t
    \right) \\ 
  &= E(t) A(-f T_x \sin\alpha, -f T_y \cos\alpha)  \\
  &= E(t) A(\overline\Delta_x \sin\alpha, \overline\Delta_y \cos\alpha)  \\
\end{split}
\end{equation}
%
dove $\overline\Delta_x = \Delta_x/\lambda$.  Ne deduciamo quindi che
il segnale viene amplificato di $\abs{A(\overline\Delta_x \sin\alpha,
\overline\Delta_y \cos\alpha)}$ e sfasato di $\angle
A(\overline\Delta_x \sin\alpha, \overline\Delta_y \cos\alpha)$
(rispetto al segnale acquisito dal microfono nell'origine).  Si
osservi che $A: \reali^2 \to \complessi$ \`e la trasformata di Fourier
del filtro bidimensionale $a_{n,m}$.  Si osservi inoltre $A$ viene
``campionata'' sull'ellisse di assi $\overline\Delta_x = f T_x$,
$\overline\Delta_y = f T_y$.   Si veda \fref{ref}

\subsection{Il caso del reticolo generico}
\label{sub:0;appunti}

Riprendiamo la \er{eq:11;appunti} e riscriviamola senza fare uso delle
coordinate $x$ e $y$

%
\begin{equation}
\label{eq:13;appunti}
\begin{split}
  s_p(t) &=  E(t) \exp\left[j2\pi  \left(\frac{x}{\lambda}
    \sin\alpha+\frac{y}{\lambda}  \cos\alpha
    \right)\right] \\ 
 &= E(t) \exp\left(j2\pi  \frac{u_\alpha^t}{\lambda} p\right) 
\end{split}
\end{equation}
%
dove $u_\alpha = [\sin\alpha, \cos\alpha]$.  Supponiamo ora che i
microfoni siano su un reticolo $\Lambda$, il risultato della somma
pesata diventa
%
\begin{equation}
\label{eq:14;appunti}
\begin{split}
  x(t) &= \sum_{p\in\Lambda} a_p E(t) \exp\left(j2\pi
  \frac{u_\alpha^t}{\lambda} p\right)  \\
       &= E(t) \sum_{p\in\Lambda} a_p \exp\left(j2\pi
  \frac{u_\alpha^t}{\lambda} p\right)  \\
       &= E(t) A(u_\alpha/\lambda) \\
       &= E(t) A(f u_\alpha/v) 
\end{split}
\end{equation}
%
dove $A : \reali^2/\Lambda^* \to \complessi$ \`e la risposta in
frequenza del filtro $a : \Lambda \to \complessi$.  Si osservi che ora
la risposta in frequenza viene campionata su un cerchio e non pi\`u
su un'ellisse.  La spiegazione \`e che nel caso precedente lo
stiracchiamento degli assi era dovuto al fatto che il filtro era
sempre su $\interi^2$, anche se $\Delta_x\ne \Delta_y$.  Ora gli assi
sono scalati secondo la geometria reale ed il ``campionamento''
avviene su una circonferenza di raggio proporzionale alla frequenza
del segnale.

% \appendix
% \section{Roba vecchia}
% \label{sect:2;appunti}
% 
% \subsection{Onda piana}
% \label{sect:0;appunti}
% 
% Sia $\bfu \in \reali^3$ un vettore unitario $\norma\bfu=1$.  Un
% segnale spazio-temporale $s : \reali^3 \times \reali \to \complessi$
% \`e un'\emph{onda piana che viaggia in direzione $\bfv$ con velocit\`a
%   $v$} se $s$ assume in $\bfx$ lo stesso valore che aveva assunto
% nell'origine $\bfu^t\bfx/v$ secondi fa, ossia
% %
% \begin{equation}
% \label{eq:0;appunti}
% \forall \bfx,t \qquad s(\bfx, t) = s(0, t-\bfu^t\bfx/v)
% \end{equation}
% %
% In altre parole, $s$ \`e un'onda piana se esiste $g : \reali \to
% \complessi$ tale che
% %
% \begin{equation}
% \label{eq:1;appunti}
% s(\bfx, t) = g(t-\bfu^t\bfx/v) = g(t-\bfk^t\bfx)
% \end{equation}
% %
% dove $\bfk=\bfu/v$.
% 
% Supponiamo ora che $g$ sia un'esponenziale complesso di frequenza $f$
% %
% \begin{equation}
% \label{eq:2;appunti}
% g(t) = \exp(j2\pi ft)
% \end{equation}
% %
% Ne segue che
% %
% \begin{equation}
%   \label{eq:3;appunti}
%   \begin{split}
% s(\bfx, t) &= \exp(j2\pi f(t-\bfk^t\bfx))\\
% &= \exp(j2\pi ft) \exp(-j2\pi f\bfk^t\bfx))\\
% &= \exp(j2\pi ft) \exp(-j2\pi (f/v)\bfu^t\bfx)) \\
% &= \exp(j2\pi ft) \exp(-j2\pi\bfu^t\bfx/\lambda))
%   \end{split}
% \end{equation}
% %
% dove $\lambda = f/v$ \`e la lunghezza d'onda.
% 
% Supponiamo ora di avere dei microfoni posizionati e che l'$n$-simo
% microfonato sia posizionato in $n L
% \bfb$, dove $n \in \interi$, $L \in\reali$ \`e la distanza tra
% microfoni e $\norma\bfb=1$.  Il segnale acquisito dall'$n$-simo
% microfono \`e quindi
% %
% \begin{equation}
%   \label{eq:4;appunti}
%   \begin{split}
% r_n(t) \perdef s(nL\bfb, t) &=
% \exp(j2\pi ft) \exp(-j2\pi n \bfu^t\bfb L/\lambda) \\
% &= \exp(j2\pi ft) \exp(-j2\pi n \cos(\theta) L/\lambda) \\
% &= \exp(j2\pi ft) \exp(-j2\pi \muth n) \\
%   \end{split}
% \end{equation}
% %
% dove $\theta$ \`e l'angolo tra $\bfu$ e $\bfb$ e $\muth = \cos(\theta)
% L/\lambda$.  Supponiamo ora di moltiplicare per $\alpha_n$ e ritardare
% di $\tau_n$ il segnale $r_n$ prima di ``impastare'' tra loro tutti i
% segnali.  Il segnale di uscita $y : \reali \to \complessi$ \`e
% %
% \begin{equation}
% \label{eq:5;appunti}
%   \begin{split}
%     y(t) &= \sum_n \alpha_n r_n(t-\tau_n) \\
%          &= \sum_n \alpha_n \exp(j2\pi ft) \exp(-j2\pi f\tau_n)
%     \exp(-j2\pi \muth n) \\ 
%     &= \exp(j2\pi ft) \sum_n a_{f}(n) \exp(-j2\pi \muth n) \\
%     &= \exp(j2\pi ft) A_f(\muth)
%   \end{split}
% \end{equation}
% %
% dove $a_f(n) \perdef \alpha_n \exp(-j2\pi f\tau_n)$ e $A_f(\muth)$ \`e
% la ``trasformata di Fourier'' di $a_f$ calcolata in $\muth =
% \cos(\theta) L/\lambda$. La notazione $a_f(n)$ \`e stata scelta
% apposta per interpretare i coefficienti $\alpha_n \exp(-j2\pi
% f\tau_n)$ come un segnale a tempo discreto difendente da $f$.
% 
% In generale $A_f(\muth)$ dipende da $f$, ma nel caso specifico in cui
% $\tau_n=\tau_0$ per ogni $n$, il fattore $\exp(-j2\pi f\tau_0)$
% pu\`o essere estratto dalla sommatoria ed ``assorbito'' come una fase
% in $\exp(j2\pi ft)$ (il che � ovvio: stiamo traslando tutti gli $r_n$
% nello stesso modo e tale traslazione si riflette sul segnale di
% uscita).
% 
% Quando $\theta$ va da $0$ a $\pi$, $\muth$ passa da $L/\lambda$ a
% $-L/\lambda$.  Se $L = \lambda/2$, vengono passate tutte le frequenze
% normalizzate da $-1/2$ a $1/2$, se $L > \lambda/2$, c'\`e aliasing,
% mentre se $L < \lambda/2$ solo un sottoinsieme delle frequenze vengono
% esplorate.
% 
% Vediamo le lunghezze d'onda tipiche del suono: a 440 Hz, e 300 m/s
% abbiamo circa 68~cm, se andiamo a 44kHz, abbiamo 7~mm circa a 50~kHz
% abbiamo 6~mm (quindi $L$ ottima \`e 3mm)
% 
% \section{Il caso multidimensionale}
% \label{sect:1;appunti}
% 
% Supponiamo di disporre i microfoni lungo un reticolo $\ret\bfM$ di
% base $\bfM \in \reali^{3\times 2}$.  Abbiamo quindi per $\bfn \in
% \interi^2$
% %
% \begin{equation}
% \label{eq:6;appunti}
% \begin{split}
%   r_{\bfn}(t) = s(\bfM\bfn, t)
%   &=
%   \exp(j2\pi ft)
%   \exp\left(-j2\pi \frac{\bfu^t\bfM}{\lambda}\bfn\right) \\
%   &=
%   \exp(j2\pi ft)
%   \exp\left(-j2\pi \bfv^t\bfn\right) \\
% \end{split}
% \end{equation}
% %
% dove
% %
% \begin{equation}
% \label{eq:7;appunti}
% \bfv = \frac{\bfM^t \bfu}{\lambda} =
% \vcii{\mu_1\cos\theta_1}
% {\mu_2\cos\theta_2}
%  \in \reali^{1\times 2}
% \end{equation}
% %
% dove $\theta_i$ \`e l'angolo tra $\bfu$ e l'$i$-simo vettore di base,
% mentre $\mu_i = L_i/\lambda$ \`e la lunghezza dell'$i$-simo vettore di
% base misurata in unit\`a di $\lambda$.
% 
% Il sengale di uscita risulta quindi essere
% %
% \begin{equation}
% \label{eq:8;appunti}
%   \begin{split}
%     y(t) &= \sum_\bfn \alpha_\bfn r_\bfn(t-\tau_\bfn) \\
%          &= \exp(j2\pi ft) \sum_\bfn \alpha_\bfn  \exp(-j2\pi f\tau_n)
%     \exp(-j2\pi\bfv^t\bfn) \\
%          &= \exp(j2\pi ft) \sum_\bfn a_\bfn
%     \exp(-j2\pi\bfv^t\bfn) \\
%     &= \exp(j2\pi ft) A(\bfv) \\
%    &= \exp(j2\pi ft) A(f \frac{\bfv_0}{v}) \\
%   \end{split}
% \end{equation}
% %
% \end{document}
\bibliographystyle{i3e}
\bibliography{biblio,di-jelena,miei,rfc,local}%,nldpre80
\end{document}



