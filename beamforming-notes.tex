\documentclass[11pt]{article}

\usepackage[pdfborder={1 1 0}, colorlinks=true]{hyperref}
\usepackage{amsmath}
\usepackage{amsbsy}
\usepackage{amssymb}
\usepackage{amsthm,amsfonts}
\usepackage{epsf,epsfig}
\usepackage{times}
\usepackage{mathptmx}
\usepackage{color}
\usepackage{nomelabel}
\usepackage[english]{babel}
\usepackage{srcltx}
\usepackage[boxed]{algorithm2e}
\usepackage{ifpdf}
%
% Standard definitions...
%
\input{def}
%
% True document
%
\newcommand{\wv}{\bfk}
\newcommand{\wvn}{\hat\wv}
\newcommand{\x}[1][0]{\ifcase #1\bfp \or\bfq\else x#1\fi}
\newcommand{\sig}[1][\x,t]{s({#1})}
\newcommand{\out}[1][t]{o({#1})}
\newcommand{\coeff}[1][\x]{a\parasenondot{#1}}
\newcommand{\Coeff}[1][\wv]{A\parasenondot{#1}}
\newcommand{\lcoeff}[1]{h\sedotelse{#1}{}{({#1}\Delta)}}
\newcommand{\Lcoeff}[1]{H\parasenondot{#1}}
\newcommand{\filt}[1][\x,nT]{a\parasenondot{#1}}
\newcommand{\Filt}[1][\wv,f]{A\parasenondot{#1}}

\title{Notes on beamforming}
\author{Riccardo Bernardini\\\email{riccardo.bernardini@uniud.it}
\\  \url{https://www.linkedin.com/in/riccardobernardini/}}

\begin{document}
\maketitle
\begin{abstract}
In this short document we study the problem of beamforming: we have an
array of \emph{isotropic} sensors (e.g., microphones or antennas)
positioned on the points of lattice $\Lambda \subset \reali^3$ and we
mix the output of the sensors with suitable weights in order to create
a \emph{virtual sensor}.  We want to analyze the dependence of the
behavior of the virtual sensor from the weight used for the mixing.
By working in the far-field approximation we show how the problem of
beamforming design can be reduced to a filter design problem.
\end{abstract}

\tableofcontents




\section{Far field approximation and plane waves}
\label{sect:0;beamforming-notes}

We want to study the problem of beamforming: given an array of
\emph{isotropic} sensors positioned on the points of lattice
$\Lambda$, we mix the output of the sensors with suitable weights in
order to create a \emph{virtual sensor}.  We want to analyze the
dependence of the behavior of the virtual sensor from the weight used
for the mixing.

We will work, for the sake of simplicity, the \emph{far field
  approximation}.  More precisely, we will suppose that there is a
source that emits complex sinusoidal waves and it is so far away that
the wave we receive can be approximated with a \emph{plane wave}. In
other words, the signal we ``see'' in position $\bfp \in\reali^3$ at
time $t \in \reali$ is
%
\begin{equation}
  \label{eq:1;beamforming-notes}
  \begin{split}
    \sig &=
    A \exp\left[-j 2\pi (\wv^t\x-ft + \phi) \right]\\
    &=C_\phi \exp\left[-j 2\pi (\wv^t\x-ft) \right]\\
    &=C_\phi \exp\left(-j 2\pi \wv^t\x \right)
    \exp\left(j 2\pi ft\right)\\
  \end{split}
\end{equation}
%
where $C_\phi = A \exp(-j2\pi \phi)$ encodes both phases and amplitude
of the signal.  Since everything we will do in the following will be
linear, we will just omit $C_\phi$ (that can be added at the end) in
order to simplify the notation.  The last line of
\er{eq:1;beamforming-notes} shows the space and time ``components''
separated. It is worth to discuss briefly the meaning of $f$ and
$\wv$.

\paragraph{Meaning of $f$ and $\wv$} By thinking
\er{eq:1;beamforming-notes} at a fixed position $\x$, it 
is immediate to recognize that $f$ is the frequency and it has units
of the inverse of time (e.g., Hertz if time is measured in seconds).
It is also immediate to recognize that $\sig$ is constant along the
planes orthogonal to \emph{wavevector} $\wv$. Indeed, if $\bfr$ is
orthogonal to $\wv$, that is, $\wv^t\bfr=0$, then
%
\begin{equation}
  \label{eq:2;beamforming-notes}
  \begin{split}
\exp\left(-j 2\pi \wv^t(\x+\bfr) \right)
&= \exp\left(-j 2\pi \wv^t\x \right)
\exp\left(-j 2\pi \wv^t\bfr \right)\\
&= \exp\left(-j 2\pi \wv^t\x \right)
  \end{split}
\end{equation}
%
  The wavelength $\lambda$, measured along the direction of $\wv$ is
  equal to $\lambda=1/\norma\wv$.  Indeed, let $\wvn = \wv/\norma\wv$
  be the versor associated to $\wv$ and suppose we move from $\x$ by a
  distance $\lambda$ along $\wvn$ arriving in $\x+\lambda\wvn$.  It
  turns out that the argument of the exponential is
%
\begin{equation}
\label{eq:3;beamforming-notes}
\begin{split}
  2\pi \wv^t(\x+\lambda\wvn)
  &=   2\pi \wv^t\x+2\pi\;\wv^t \lambda\wvn\\
  &=   2\pi \wv^t\x+2\pi\wv^t \commenta{\frac{1}{\norma\wv}}{\lambda}
  \commenta{\frac{\wv}{\norma\wv}}{\wvn} \\
  &=   2\pi \wv^t\x+2\pi \lambda\frac{\wv^t\wv}{\norma\wv^2}\\
  &=   2\pi \wv^t\x+2\pi
\end{split}
\end{equation}
%
That is, by moving along $\wvn$ by a distance of $\lambda=1/\norma\wv$, the
argument of the exponential changes by $2\pi$, showing that we moved
by a wavelength.  Therefore, \emph{wavevector $\wv$ encodes both
  direction and wavelength}.


\paragraph{Relationship between frequency $f$ and wavevector $\wv$}
As well known, the velocity $v$ of a wave can be written as $v=\lambda
f$. In many cases, physical waves have a fixed velocity (e.g., sound,
light), therefore,   $\lambda$ (and, consequently, the length
of $\wv$) depends on $f$.  More precisely, one can write
%
\begin{equation}
\label{eq:4;beamforming-notes}
\wv = \wvn\; \frac{f}{v}
\end{equation}
%
that is, if velocity $v$ is fixed, the length of the wavevector $\wv$ is
directly proportional to frequency $f$.

\section{Beamforming}
\label{sect:1;beamforming-notes}

Suppose that we have \emph{isotropic} sensors placed on the points of
a lattice $\Lambda \subset \reali^3$.  We combine the output of the
sensors by weighting the output of the sensor in $\x \in \Lambda$
with weight $\coeff\in\complessi$ and adding the result.

\begin{commento}
  Note that this setup can be used even if the sensors are
  irregularly spaced, as long as the set of positions $S \subset\reali^3$ where
  sensors are placed is the subset of a suitable lattice $\Lambda$: we
  will just assign zero weight to the lattice locations that have no
  sensor.

  The hypothesis that $S$ is a subset of some lattice is quite
  ``cheap.'' Indeed, it is easy to show that it suffices that the
  coordinates of the points in $S$ are rationally related, that is, it
  is true that
  %
\begin{equation}
\label{eq:0;beamforming-notes}
\forall p,q \in S,  n \in \{1, 2, 3\}
\quad p_n=0 \vee q_n/p_n \in \razionali
\end{equation}
%
If $S$ does not satisfy \er{eq:0;beamforming-notes}, it suffices to
change its elements a tiny bit (that can be as small as desired) to
make \er{eq:0;beamforming-notes} true.  Therefore, from a practical
viewpoint, we can always suppose \er{eq:0;beamforming-notes} without
much loss of generality.
\end{commento}

Remembering that $\sig$ is the signal at position $\x$ at time $t$,
the output at time $t$ of the virtual sensor is
%
\begin{equation}
\label{eq:5;beamforming-notes}
\out = \sum_{\x \in \Lambda} \coeff \sig
\end{equation}
%
that, remembering \er{eq:1;beamforming-notes}, gives
%
\begin{equation}
\label{eq:6;beamforming-notes}
\begin{split}
  \out &= \sum_{\x \in \Lambda} \coeff \exp\left(-j 2\pi \wv^t\x \right)
    \exp\left(j 2\pi ft\right)\\
   &= \exp\left(j 2\pi ft\right) \sum_{\x \in \Lambda} \coeff
    \exp\left(-j 2\pi \wv^t\x \right) 
    \\
    &= \exp\left(j 2\pi ft\right) \Coeff \\
    &= \exp\left(j 2\pi ft\right) \Coeff[.]\left(\frac{f\;\wvn}{v} \right)
\end{split}
\end{equation}
%
In other words, the output of the weighted sum is a complex
exponential at frequency $f$ multiplied by $\Coeff$, the Fourier
transform of $\coeff[.]$ evaluated in $\wv$. The last line of
\er{eq:6;beamforming-notes} show explicitely the dependence on the
direction versor $\wvn$ and the frequency $f$.

Result \er{eq:6;beamforming-notes} is interesting because it reduces
the problem of determining coefficients $\coeff$ to the problem of
designing a 3-dimensional filter with a desired frequency response.

\subsection{The 1-dimensional case: sensor array}
\label{sub:1.0;beamforming-notes}

A case that is quite common is when the sensor are on a single line.
Without loss of generality, we can think that they are aligned along
the first axis at distance $\Delta$.  A suitable lattice is the
orthogonal lattice
%
\begin{equation}
\label{eq:12;beamforming-notes}
\Lambda=\interi^3(\Delta) = \left\{\vciii{n_1\Delta}{ n_2\Delta}{
  n_3\Delta} |  \; n_1,
n_2, n_3 \in \interi\right\}
\end{equation}
%
Let
$\lcoeff n$ be the coefficient of the sensor placed in position
$n\Delta$ along the first axis, that is, in $[n\Delta, 0,
  0]$. Coefficient $\coeff$, $\x=[ n_1\Delta, n_2\Delta, n_3\Delta]
\in \Lambda$ can be written as
%
\begin{equation}
\label{eq:7;beamforming-notes}
\coeff[n_1\Delta, n_2\Delta, n_3\Delta] = \lcoeff{n_1} \delta(n_2) \delta(n_3)
\end{equation}
%
that is, ``virtual'' signal $\coeff[.] : \Lambda \to \complessi$ is
separable.  It turns out that $\Coeff[.] :
\reali^3/\Lambda^*\to\complessi$ is 
separable too and that 
%
\begin{equation}
\label{eq:8;beamforming-notes}
\Coeff[k_1, k_2, k_3] = \Lcoeff{k_1}
\qquad \wv \in \reali^3/\Lambda^*
\end{equation}
%
Note that $\Lambda^*=\interi^3(\inv\Delta)$, therefore $k_1 \in
\reali/\interi(\inv\Delta)$. By expressing $\wv$ in a suitable spherical
coordinate system\footnote{$\theta$ is the angle between $\wv$ and the
  first axis, $\phi$ is the angle between the projection of $\wv$ on
  the plane of the other two axes and the second axis.}
  so that
%
\begin{equation}
\label{eq:10;beamforming-notes}
\wv=\norma\wv \vciii{\cos\theta}
  {\sin\theta\cos\phi}{\sin\theta\sin\phi}
\end{equation}
%
we get, remembering that $\norm\wv=f/v$,
%
\begin{equation}
\label{eq:9;beamforming-notes}
\Coeff[.]\left(\frac{f}{v}\wvn \right)
= \Lcoeff.\left(\frac{f}{v}\cos\theta \right)
\end{equation}
%
It turns out that at a given frequency $f$ the sensitivity of the
virtual sensor depends only on the angle $\theta$ between $\wv$ and
the first axis. Moreover, the sensitivity as function of $\theta$ is
the frequency response of $\lcoeff.$ ``distorted'' by the cosine. If
we have a desired receiving pattern, \er{eq:9;beamforming-notes}
reduces the problem to finding coefficients $\lcoeff n$ to the problem
of designing a FIR filter with a given frequency response.

\begin{commento}
  Frequency response $\Lcoeff.$ is periodic with period
  $\inv\Delta$. If $f/v$ is too large, $(f/v) \cos \theta$ can span
  more than one period, giving rise to a sensitivity that is
  ``repetitive'' (not truly periodic because of the cosine) in
  $\theta$.  
  
  If this repetition is not desired, $(f/v)\cos\theta= (1/\lambda)
  \cos\theta$ must remains between $\pm \inv\Delta/2$ for every
  $\theta$.  This is true if and only if
  %
\begin{equation}
\label{eq:11;beamforming-notes}
\frac{1}{\lambda} < \frac{\inv\Delta}{2}
\quad \sse \quad
\Delta < \frac \lambda 2
\end{equation}
%
that is, the space between the sensor must be smaller than half the
wavelength. 
\end{commento}

\section{Adding filtering}
\label{sect:2;beamforming-notes}

Let's make the system a bit more complex: instead of just multiplying
every sensor output by a weight, let's filter the sampled version of
the output of the sensor in $\x$ before mixing.  Let $T$ be the
sampling interval and denote with $\filt$ the $n$-th sample of the
impulse response associated with the sensor in $\x$.  Note that now
$\filt[.]$ can be considered as signal defined on lattice $\Lambda
\times \interi(T)$.  The output of the mixer at time $mT$ is
%
\begin{equation}
\label{eq:13;beamforming-notes}
\begin{split}
  \out[mT]
  &= \sum_{\x \in\Lambda} \sum_{n \in \interi}  \filt
  \exp\left(-j 2\pi \wv^t\x \right)
    \exp\left(j 2\pi f(mT-nT)\right)\\
  &= \sum_{\x \in\Lambda} \sum_{n \in \interi}  \filt
  \exp\left(-j 2\pi \wv^t\x \right)
    \exp\left(j 2\pi fmT\right)
    \exp\left(-j 2\pi fnT\right)\\
  &=     \exp\left(j 2\pi fmT\right)
\sum_{\x \in\Lambda} \sum_{n \in \interi}  \filt
  \exp\left(-j 2\pi \wv^t\x \right)
    \exp\left(-j 2\pi fnT\right) \\
    &=     \exp\left(j 2\pi fmT\right) \Filt\\
    &=     \exp\left(j 2\pi fmT\right) \Filt[.]\left(\frac{f}{v}\wvn,f \right)
\end{split}
\end{equation}
%
where $\Filt[.] : \reali^4/(\Lambda^* \times \interi(1/T)) \to
\complessi$ is the Fourier transform of $\filt[.] : \Lambda \times
\interi(T) \to \complessi$.




\bibliographystyle{i3e}
\bibliography{biblio,di-jelena,miei,rfc,local}%,nldpre80
\end{document}




%%  LocalWords:  beamforming eq wavevector versor biblio di jelena
%%  LocalWords:  miei rfc
